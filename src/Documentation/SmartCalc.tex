\documentclass{article}
\usepackage[a4paper, total={6in, 8in}]{geometry}
\title{SmartCalc - is engineering calculator}
\author{harmonic}
\begin{document}
\maketitle
This application will help you calculate complex expressions based on the priorities of operations.
SmartCalc supports both unary and binary operations. Parentheses are used to prioritize operations: $ ( -a ) + b  or  - ( a + b ) $.
SmartCalc performs the calculation after clicking on " = " - symb. 
In SmartCalc, you can use both integers and floating-point numbers for calculations. A period " . " is used instead of a comma " , ".

\textbf{List of supported binary operations:}
\begin{enumerate}
\item Addition: $a + b$;
\item Subtraction: $a - b$;
\item Multiplication: $a * b$;
\item Division: $a / b$;
\item Power: $a ^ b$;
\item Modulus: a mod b;
\end{enumerate}

\textbf{List of supported unary operations:}
\begin{enumerate}
\item Unary plus;
\item Unary minus;
\end{enumerate}

\textbf{With the help of a SmartCalc, you can plot the function:}
\begin{enumerate}
\item cos(x);
\item sin(x);
\item tan(x);
\item acos(x);
\item asin(x);
\item atan(x);
\item sqrt(x);
\item ln(x);
\item log(x);
\end{enumerate}

To plot function graphs, you need to use "kX" and specify the maximum values of x, y and step.
kX and y are limited to at least numbers from -1000000 to 1000000.
Verifiable accuracy of the fractional part is at least to 7 decimal places.
Users must be able to enter up to 255 characters
\end{document}